\documentclass{article}

\usepackage[colorlinks=true,allcolors=blue]{hyperref}
\usepackage{booktabs}
\usepackage{framed}

\title{Taylor University\\
  Faculty Mentored Undergraduate\\
  Summer Scholarship Program\\
  Faculty Application}

\author{Tom Nurkkala and Ken Kiers}

% Must accompany student applications.

% Faculty Mentored Scholarship is defined as inquiry, investigation and creative
% activity conducted by faculty and students together to make an original
% intellectual or creative contribution to the discipline in ways commensurate
% with the standards, methodologies and peer-review expectations of the field.

% Award requirements:

% 1. If awarded, a progress report from the faculty member is required.  Reports
% will be provided to the Office of Sponsored Programs by August 31, 2016.

% 2. Participation in summer enrichment activities is expected and encouraged.  A
% schedule of activities will be made available during the spring semester.

% 3. Allow time in schedule for students to attend the RCR discussion group as
% well as summer scholarship lunches.

% Completed applications are due in the Sponsored Programs Office on or before
% February 1, 2016.  Decisions will be made by March 15, 2016.

\begin{document}

\maketitle

\section{Participants}
\label{sec:participants}

The co-principal investigators on the project are:
\begin{itemize}
\item Dr.\ Tom Nurkkala\\
  Associate Professor, Computer Science and Engineering\\
  \texttt{tnurkkala@cse.taylor.edu}
\item Dr.\ Ken Kiers\\
  Professor and Chair, Physics and Engineering\\
  \texttt{knkiers@taylor.edu}
\end{itemize}

Students involved in the project are:
  \begin{itemize}
  \item TBD
  \item TBD
  \end{itemize}

% Please type or print clearly your answer to each of the questions
% below. Answers may be provided on a separate sheet: if so, please retype the
% questions on the attached sheets.

\section{Proposed Project}
\label{sec:proposed-project}

% Please describe your proposed project.

Current and emerging learning management systems
represent a considerable improvement on the paper grade book,
but suffer from two key limitations.
First, they provide little actionable information
that can be employed directly
to engage students,
enhance learning outcomes,
and simplify course management.
Second, they emphasize massive scalability
at the expense of
personalized student-teacher engagement.

Just as modern business intelligence software
distills raw data into action\-able management information,
the next generation of learning management software
should provide actionable educational intelligence
that supports continuous, measurable improvements
in teacher productivity and student outcomes.
Furthermore, next generation learning management systems
should re-personalize the relationship between teacher and student
into a modern apprenticeship model that
encourages mastery learning without sacrificing scalability
to larger populations of students.

\subsection{Actionable}
\label{sec:actionable}

Prior to the 1979 introduction of
Visicalc---the first computerized spreadsheet program---business people
used pencils and paper spreadsheets,
pre-printed with the now-familiar rows and columns of cells.
Although Visicalc and its offspring represented a huge advance over paper,
even modern spreadsheet software does little more than automate the manual
calculations required by its paper progenitors.
By the 1990's,
business had gone beyond the spreadsheet,
embracing a large class of more sophisticated
systems known collectively as Business Intelligence (BI) software.
While electronic spreadsheets remain indispensable tools for organizing \emph{raw data},
BI systems transform data into \emph{actionable information}
that can be used by business people
to analyze, manage, and empower their organization.

In educational technology,
the computerized learning management system (LMS)
represents an advance over the paper grade book
that is analogous to Visicalc's superiority over the paper spreadsheet.
In addition to tracking scores and grades,
systems like Blackboard and Moodle
are adept at distributing content,
tracking homework submissions,
mediating on-line ``conversations,''
and otherwise modernizing the learning environment.
But also like the electronic spreadsheet,
the current generation of LMS does not go much beyond organizing \emph{raw data}.
In particular, the present-day LMS does not provide either the student or the teacher
with significant \emph{actionable information} that helps
analyze, manage, or empower the learning enterprise.

\subsection{Personalized}
\label{sec:personalized}

Current trends in educational technology
exhibit breathless enthusiasm for scaling education
to the very large.
Distance learning,
on-line classes and degree programs,
and the ominously named Massive Open Online Course (MOOC) model
make this tendency evident.
Examples abound
from non-profit initiatives like EdX and Khan Academy,
and from for-profit companies like Coursera and Udacity,
not to mention the central place of technology in the business models of
for-profit institutions like Walden and the University of Phoenix.

The emphasis on massively scalable educational opportunity
is understandable in an age of
financial challenges in education,
large numbers of non-tradi\-tional students,
ubiquitous mobile tech\-nology,
global Internet access,
and billion-member online social networks.
But in the headlong rush to deliver pre-packaged content to the masses,
we have neglected the historically significant relationship
between teacher and student that has been the hallmark
of deep learning since Socrates himself.
Significant, impactful, and long-lasting educational
outcomes are surely more likely to spring
from a face-to-face mentoring relationship than
from a YouTube video or an on-line chat room.

\subsection{Learning Management 2.0}
\label{sec:lms-2.0}

The next generation of learning management system, then,
should exhibit these interlocking characteristics:
\begin{enumerate}
\item \emph{Actionable}. It should distill raw learning data
  into educational intelligence on which students
  and (especially) teachers can act.
\item \emph{Personalized}. It should make explicit
  the learning needs of individual students,
  informing teachers in real time of the need for remediation
  or the opportunity for enhanced learning.
\item \emph{Scalable}. It should scale to a reasonably large numbers of students
  without unnecessarily burdening the teacher
  or compromising the foregoing characteristics.
\end{enumerate}

As a unifying motif for the next generation of learning management system,
I propose a time-honored system of mastery learning: apprenticeship.
In the medieval guild system,
a teenager was apprenticed to a master craftsman
to gain knowledge and develop skills through
direct exposure,
extensive practice,
constant vigilance,
and immediate feedback.
Apprenticeship was the essence of \emph{personalized} education then,
and remains a viable model for effective education today.
This is not only true of vocational and technical schools.
Indeed, many undergraduate programs require
industry practica (a kind of ``apprenticeship light''),
and graduate education is itself the clearest modern example of apprenticeship
in higher education.

Apprenticeship is highly personalized,
but its one-to-one ``faculty-student ratio''
appears to make it a hopelessly impractical anachronism.
Enter the next generation of learning management system
as a key enabling technology.
It can relieve teachers
of the mundane and time consuming details
of course management.
It can provide actionable educational intelligence to teachers,
encouraging timely, strategic, and personal connection with students.
And it can scale to a large number of students.
Consider the following examples of capabilities such a system could deliver.

\begin{enumerate}
\item{Course Management}
  \begin{enumerate}
  \item Rich tools for course planning and syllabus development, allowing topics and
    activities to be aligned explicitly with course learning outcomes and categorized
    according to Bloom's taxonomy.
  \item Decoupling of course planning from course scheduling, allowing existing courses to
    be scheduled in future semesters with ease.
  \item Automated production of the final syllabus based on course planning and scheduling
    information.
  \item Topical overview prepared automatically for each class session, showing students the
    day's topics, preparatory readings, follow-up assignments, upcoming exams, etc.
    Daily topics would be tied explicitly to course learning outcomes
    to contextualize the day's learning.
  \end{enumerate}
\item{Classroom Experience}
  \begin{enumerate}
  \item Teacher and students make full use of mobile technology in the classroom,
    coordinated by interaction with a central server and database.
  \item Automatic attendance taken when students ``log in to the class.''
  \item Speaker console decoupled from classroom display,
    encouraging meaningful classroom visuals,
    avoiding ``death by PowerPoint,'' and
    discouraging the teacher's use of a slide deck as teleprompter.
  \item Ubiquitous use of mobile devices for classroom interaction; for example,
    in-class quizzing (``clickers on steroids''),
    mobile-based exercises by individual students or small student groups,
    real time monitoring by teacher of any student's work in progress,
    display of selected student work for classroom display and discussion.
  \item Efficacy of teaching strategies measured empirically using in-class quizzing.
  \item Student control of classroom graphics on personal mobile device,
    allowing review of material without disrupting class and
    consuming material at an individualized pace.
  \item Server-based tracking of the pace of student engagement with course material
    throughout a class.
  \item Student submission of private comments, requests, suggestions, or reminders to
    teacher for attention either during or after class.
  \end{enumerate}
\item{Individualized Mentoring}
  \begin{enumerate}
  \item Invitations to office hours (or other remediation) for students who are lagging
    behind the class (e.g., as measured by whether they ``keep up'' when viewing classroom
    graphics as they are covered by the teacher).
  \item Remedial exercises for students struggling with in-class exercises or quizzes.
  \item Relief from assigned homework for students demonstrating excellent performance on
    in-class exercises.
  \item Referral to campus academic enrichment program for students demonstrating possible
    learning challenges.
  \end{enumerate}
\item{Educational Analytics}
  \begin{enumerate}
  \item Course ``dashboard'' for teachers that distills raw data on attendance, in-class
    quiz results, students in need of personal attention, etc.
  \item Graphical time line showing which students are ``keeping up'' with class
    material, which students may need more attention, which are asleep, etc.
  \item Analysis of time spent on each topic, tied to course planning information to help
    ensure class is on schedule and that schedules are realistic.
  \item Information to help optimize the course for future semesters
    (e.g., content, pacing, student interaction, exercise relevance).
  \item Statistical analysis and graphical display of efficacy of in-class learning
    exercises, quiz results, etc.
  \end{enumerate}
\end{enumerate}

None of these capabilities require the development of radically new technologies.
The principle intellectual challenge is to assemble existing technologies
in ways that facilitate this actionable, personalized, and scalable vision for
teaching and learning.
The principle need that must be met to make this vision a reality
is adequate funding for a collaborate team with expertise in
computer science,
software engineering,
user experience design,
curriculum development,
and information technology.

\section{Expertise and Scholarly Plans}
\label{sec:expertise-plans}

% Please describe your general areas of expertise
% and how the proposed project fits
% into your overall scholarship plans.

TN has extensive background as a software engineer
and software engineering manager
in industry and the academy.
He has supervised numerous student software projects
for missions partners including
Operation Mobilization (England, Hong Kong),
Wycliffe/SIL (Thailand),
Global Recordings Network (Australia),
and
Tiny Hands International (Nepal).
TN's scholarly focus is on the design and development
of software tools and computational infrastructure
that enhances student engagement
and facilitates faculty-student mentoring
by providing actionable insight
into student behavior and academic performance.

\textbf{Needs to be updated}
KK is a theoretical physicist by training but has a certain amount
of background as a computer programmer.  He has developed much of
the code base for iGrad and iChair over the past 3-1/2 years
as a response to perceived needs within his department.  KK's
background as a department chair gives him a perspective that would
be useful to EdBrite and to the project as a whole.  KK's main
scholarship emphasis will likely continue to be in theoretical
particle physics, but he has a strong interest in making the
functionality of iGrad and iChair more widely available to
users beyond Taylor.

\section{Student Involvement}
\label{sec:student-involvement}

% Please describe the project in which the student will be involved.

% To the extent possible, describe what you anticipate will be the student’s
% responsibilities and activities contributing to the completion of this
% project as well as professional venues available for students to present
% their work.

\section{Student Interaction}
\label{sec:student-interaction}

% Please describe how you personally plan to interact with the students.

% Please note:

% (1) Mentors will be available on campus and/or at the same field site for a
% minimum of four hours/day for 80\% of the project days.  On those days when the
% faculty mentor is not physically on-site, he/she will arrange supervisory
% coverage with another faculty member. Covering faculty will know where students
% are supposed to be and what they are supposed to be doing. Students must have
% phone access to their mentor and/or coverage supervisor.  Mentors must train
% students on or allow time for students to complete university training for
% Responsible Conduct of Research, laboratory orientation, other pertinent safety
% protocols including chemical hygiene, and provide campus safety contact info.

% (2) FMUSS faculty are expected to plan a program for all summer students to
% provide opportunities for scholarly updates, special speakers and/or field
% trips.

\section{Resources Available}
\label{sec:resources-available}

% Please list grant or other resources available to support this work (indicate
% sources and amount).

% Include Women’s Giving Circle, SRTP, or external funding.  Do you have other
% funds to support your summer salary?  Are you requesting a summer stipend
% from the Provost’s Fund?

We hope that faculty salaries will be supplemented
by funds available from
an ``indirect costs'' account associated with KK's research.

TN has had a preliminary discussion with a donor
who may be willing to fund students who are
participating in a missions computing project
such as this.

\section{Project Expenses}
\label{sec:project-expenses}

% Please list anticipated project expenses and how resources will be allocated
% to cover expenses.

% A small supplies grant (\$300 - \$500) may be awarded with each FMUSS program
% award.

We request the full stipend amount available for
two faculty members
and two students, as follows.

\begin{center}
  \begin{tabular}{rlrr} \toprule
    Qty & Description    & Unit    & Total    \\ \midrule
    2   & Faculty salary & \$3,200 & \$6,400  \\
    2   & Student salary & 3,200   & 6,400    \\ \midrule
        & Total          &         & \$12,800 \\ \bottomrule
  \end{tabular}
\end{center}

No additional funds would be required for other expenses.
% {\bf (Tom: Would we need to include funds for field trips, speakers, scholarly updates, etc.?)}

\section{Facilities and Equipment}
\label{sec:facilities-equipment}

% Please list any rooms, labs, or major instruments that will be required
% during the course of this project.

The project participants will work in
the Center for Missions Computing lab (Euler~040),
the Computer Science and Engineering lab (Euler~217), and
faculty offices.
The project will not require the use of major instruments
(other than existing computers).

\section{Back-up Supervision}
\label{sec:back-up-supervision}

% Please list college employees that will be requested as supervisors during
% any leaves of absence.

Faculty members plan to coordinate absences
so that one of them will be present with students if the other is absent.
If both faculty members are absent,
back-up supervision will be provided by
Mr.\ Nate White,
Dr.\ Stefan Brandle, or
Dr.\ Jonathan Geisler
in regular consultation with the faculty members.

\end{document}

%%% Local Variables:
%%% mode: latex
%%% TeX-master: t
%%% End:
