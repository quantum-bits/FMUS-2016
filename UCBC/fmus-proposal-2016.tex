\documentclass{article}

\usepackage[colorlinks=true,allcolors=blue]{hyperref}
\usepackage{booktabs}
\usepackage{framed}

\title{Taylor University\\
  Faculty Mentored Undergraduate\\
  Summer Scholarship Program\\
  Faculty Application}

\author{Ken Kiers and Tom Nurkkala}

\newcommand{\igrad}{\texttt{iGrad}}
\newcommand{\ichair}{\texttt{iChair}}
\newcommand{\edbrite}{\texttt{EdBrite}}

% Must accompany student applications.

% Faculty Mentored Scholarship is defined as inquiry, investigation and creative
% activity conducted by faculty and students together to make an original
% intellectual or creative contribution to the discipline in ways commensurate
% with the standards, methodologies and peer-review expectations of the field.

% Award requirements:

% 1. If awarded, a progress report from the faculty member is required.  Reports
% will be provided to the Office of Sponsored Programs by August 31, 2016.

% 2. Participation in summer enrichment activities is expected and encouraged.  A
% schedule of activities will be made available during the spring semester.

% 3. Allow time in schedule for students to attend the RCR discussion group as
% well as summer scholarship lunches.

% Completed applications are due in the Sponsored Programs Office on or before
% February 1, 2016.  Decisions will be made by March 15, 2016.

\begin{document}

\maketitle
\tableofcontents
\newpage

\section{Participants}
\label{sec:participants}

The co-principal investigators on the project are:
\begin{itemize}
\item Dr.\ Ken Kiers\\
  Professor and Chair, Physics and Engineering\\
  \texttt{knkiers@taylor.edu}
\item Dr.\ Tom Nurkkala\\
  Associate Professor, Computer Science and Engineering\\
  \texttt{tnurkkala@cse.taylor.edu}
\end{itemize}

Students involved in the project are:
  \begin{itemize}
  \item TBD
  \item TBD
  \end{itemize}

% Please type or print clearly your answer to each of the questions
% below. Answers may be provided on a separate sheet: if so, please retype the
% questions on the attached sheets.

\section{Proposed Project}
\label{sec:proposed-project}

% Please describe your proposed project.

This section provides project background information
and proposed project activities.
We also reflect on the project's relevance
to the Boyer model of scholarship.

\subsection{Background}
\label{sec:background}

Universit\'e Chr\'etienne Bilingue du Congo (UCBC)
is a fledgling Christian university located in Beni, Congo.
UCBC faces significant challenges in information technology (IT)---specifically
lack of support for a computerized
student information system (SIS) or
learning management system (LMS).\footnote{For reference,
  Taylor's SIS is \emph{Banner}
  and its LMSs are \emph{Blackboard} and \emph{Moodle}.}

Staff from UCBC first contacted Taylor several years ago to learn
whether the Taylor Center for Missions Computing (CMC) could help
address their IT needs.  During a visit to Taylor's campus in fall
2015, UCBC representatives learned about \ichair{} and \igrad{}, two
software packages developed by KK, TN, and several students at Taylor.

\igrad{} is a web-based application that can be used by undergraduate
students to develop four-year plans and conduct graduation audits on
those plans to ensure that graduation requirements will be met.  The
application was developed approximately three years ago by TU faculty
and students and is used extensively in the Physics and Engineering
Department as an advising tool.  The functionality of \igrad{} is
similar to that provided by DegreeWorks.

\ichair{} is a web-based application that aids department chairs in
the planning and visualization of faculty schedules and loads.  This
package was developed by KK, with significant help from TN.  Several
departments at Taylor currently use \ichair{} as a planning tool.

We conducted several follow-up conversations
with UCBC staff in Congo,
including a demonstration
of \igrad{} and \ichair{}
by video conference.
As we learned more about UCBC's current infrastructure,
it became clear that UCBC's needs
were even more basic than what \igrad{} and \ichair{} provided.
For example, UCBC currently operates
without computerized support
(beyond simple spreadsheets) for
admissions, student grades, transcripts,
or even rudimentary human resources needs.\footnote{In a conversation with UCBC
  several years ago, they reported that their biggest record-keeping challenge
  was that rats tended to eat the paper records. They were not joking.}

Recognizing that the functions of \igrad{} and \ichair{}
were not the most critical to UCBC's infrastructure,
we began to search for other software solutions
that might better help satisfy UCBC's immediate requirements.
Through a reference from the Dr.\ Andrew Sears, president
of City Vision University,\footnote{\url{http://www.cityvision.edu/}}
we contacted Dr.\ Walker Tzeng,
executive director of
the  World Evangelical Theological Institute Association
(WETIA).\footnote{\url{http://wetia.org/}}
WETIA and its partners have created \edbrite,
a cloud-based web application described as
\begin{quote}
  an integrated Learning Management System (LMS)
  and Student Information System (SIS) platform
  for small and medium-sized colleges, universities, schools, seminaries,
  churches, ministries, non-profits, organizations,
  and individual teachers.\footnote{\url{http://www.edbrite.com/}}
\end{quote}
Among other features, \edbrite{} supports
enrollment management,
registration,
syllabi,
announcements,
quizzes,
homework,
discussions,
reporting,
and grading.
In short,
\edbrite{} provides many of
the SIS and LMS capabilities
most needed by UCBC.

\subsection{Activities}
\label{sec:activities}

The proposed project centers on two activities:
on-site deployment in Africa,
and extending \edbrite{} with degree planning and auditing features.

\subsection{On-Site Deployment in Africa}
\label{sec:on-site-deployment}

The \edbrite{} system is currently available only as a cloud-based web application.
While unremarkable for \edbrite's existing users,
its ``on-line only'' nature is problematic for UCBC.
Facing regular outages of electric power,
compounded by slow and intermittent
Internet connectivity in Congo,
UCBC is reluctant to rely on mission-critical software
available only via the Internet.

In partnership with \edbrite's developer,
we propose to package \edbrite{} 
for deployment in Africa
by leveraging the latest software container technology.\footnote{\url{https://www.docker.com/}}
Packaged in this way,
\edbrite{} can be made available
on UCBC's internal network in Congo,
where the software will be installed, maintained, and updated
by UCBC's local IT staff.
Running on their local network eliminates UCBC's reliance on Internet availability
in order to operate the software.
In addition, UCBC has local electrical generation capacity,
which will maintain access to the software
regardless of the state of the power grid.

\subsection{Degree Planning and Auditing Feature}
\label{sec:planning-auditing}

... Re-implementing \igrad{} inside \edbrite{}.


\subsection{Scholarship}
\label{sec:scholarship}

\emph{TODO---TODO---TODO---TODO}
\begin{enumerate}
\item Boyer model stuff
\item Emphasize novel issues: alternative scheduling with no fixed academic
  calendar, re-deployment of cloud-based application to local server
  environment using container technology
\end{enumerate}

\begin{framed}
  \textbf{From Bill's E-mail}

  The NSF document lists Research, Applied Research, and Development.
  I believe what you propose can be stated so as to fit in the Development definition.
  I also assume that this could qualify for our scholarship area of application based on Boyer.
  ``Development is the systematic use of the knowledge or understanding
  gained from research
  directed toward the production of useful materials, devices, or methods,
  including the design and development of prototypes and processes.''

  Students can be viewed as doing ``research'' into issues and solutions
  for such problems in general and
  if they emphasize development of a process/prototype of use in other areas
  (including cultural, technological access and ability, etc.),
  the project could fit a TU compatible definition of scholarship.
  I believe Jeff will accept this concept of scholarship
  that is broader than traditional scientific research.
  Including concepts such as \emph{refactoring}, \emph{loose coupling with tight cohesion} might be useful.
  Fancy buzzwords shouldn't hurt!

  As you said when we talked, this project would apply knowledge to ``consequential
  problems'' that fit with the University mission and hold the promise of being
  generalizable.
\end{framed}


\section{Expertise and Scholarly Plans}
\label{sec:expertise-plans}

% Please describe your general areas of expertise
% and how the proposed project fits
% into your overall scholarship plans.

TN is the founding director of the
Taylor Center for Missions Computing,
which is dedicated to helping meet the computing needs
of Christian missions around the globe.
He has extensive background as a software engineer
and software engineering manager
in industry and the academy.
He has supervised numerous student software projects
for missions partners including
Operation Mobilization (England, Hong Kong),
Wycliffe/SIL (Thailand),
Global Recordings Network (Australia),
and
Tiny Hands International (Nepal).
TN's scholarly focus is on the design and development
of software tools and computational infrastructure
that enhances student engagement
and facilitates faculty-student mentoring
by providing actionable insight
into student behavior and academic performance.

KK is a theoretical physicist by training but has a certain amount
of background as a computer programmer.  He has developed much of
the code base for \igrad{} and \ichair{} over the past 3-1/2 years
as a response to perceived needs within his department.  KK's
background as a department chair gives him a perspective that would
be useful to \edbrite and to the project as a whole.  KK's main
scholarship emphasis will likely continue to be in theoretical
particle physics, but he has a strong interest in making the
functionality of \igrad{} and \ichair{} more widely available to
users beyond Taylor.

\section{Student Involvement}
\label{sec:student-involvement}

% Please describe the project in which the student will be involved.

% To the extent possible, describe what you anticipate will be the student’s
% responsibilities and activities contributing to the completion of this
% project as well as professional venues available for students to present
% their work.

\section{Student Interaction}
\label{sec:student-interaction}

% Please describe how you personally plan to interact with the students.

% Please note:

% (1) Mentors will be available on campus and/or at the same field site for a
% minimum of four hours/day for 80\% of the project days.  On those days when the
% faculty mentor is not physically on-site, he/she will arrange supervisory
% coverage with another faculty member. Covering faculty will know where students
% are supposed to be and what they are supposed to be doing. Students must have
% phone access to their mentor and/or coverage supervisor.  Mentors must train
% students on or allow time for students to complete university training for
% Responsible Conduct of Research, laboratory orientation, other pertinent safety
% protocols including chemical hygiene, and provide campus safety contact info.

% (2) FMUSS faculty are expected to plan a program for all summer students to
% provide opportunities for scholarly updates, special speakers and/or field
% trips.

\section{Resources Available}
\label{sec:resources-available}

% Please list grant or other resources available to support this work (indicate
% sources and amount).

% Include Women’s Giving Circle, SRTP, or external funding.  Do you have other
% funds to support your summer salary?  Are you requesting a summer stipend
% from the Provost’s Fund?

We hope that faculty salaries will be supplemented
by funds available from
(i) an ``indirect costs'' account associated with KK's research
and/or
(ii) the Taylor Center for Missions Computing.

TN has had a preliminary discussion with a donor
who may be willing to fund students who are
participating in a missions computing project
such as this.

\section{Project Expenses}
\label{sec:project-expenses}

% Please list anticipated project expenses and how resources will be allocated
% to cover expenses.

% A small supplies grant (\$300 - \$500) may be awarded with each FMUSS program
% award.

We request the full stipend amount available for
two faculty members
and two students, as follows.

\begin{center}
  \begin{tabular}{rlrr} \toprule
    Qty & Description    & Unit    & Total    \\ \midrule
    2   & Faculty salary & \$3,200 & \$6,400  \\
    2   & Student salary & 3,200   & 6,400    \\ \midrule
        & Total          &         & \$12,800 \\ \bottomrule
  \end{tabular}
\end{center}

No additional funds would be required for other expenses.
% {\bf (Tom: Would we need to include funds for field trips, speakers, scholarly updates, etc.?)}

\section{Facilities and Equipment}
\label{sec:facilities-equipment}

% Please list any rooms, labs, or major instruments that will be required
% during the course of this project.

The project participants will work in
the Center for Missions Computing lab (Euler~040),
the Computer Science and Engineering lab (Euler~217), and
faculty offices.
The project will not require the use of major instruments
(other than existing computers).

\section{Back-up Supervision}
\label{sec:back-up-supervision}

% Please list college employees that will be requested as supervisors during
% any leaves of absence.

Faculty members plan to coordinate absences
so that one of them will be present with students if the other is absent.
If both faculty members are absent,
back-up supervision will be provided by
Mr.\ Nate White,
Dr.\ Stefan Brandle, or
Dr.\ Jonathan Geisler
in regular consultation with the faculty members.

\end{document}

%%% Local Variables:
%%% mode: latex
%%% TeX-master: t
%%% End:

%  LocalWords:  TBD Universit Chr etienne Bilingue du UCBC Beni LMS LMSs KK
%  LocalWords:  DegreeWorks UCBC's Tzeng WETIA TODO generalizable TN's KK's
%  LocalWords:  rlrr
