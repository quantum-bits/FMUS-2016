\documentclass{article}

\usepackage[margin=1.5in]{geometry}
\usepackage[colorlinks=true,allcolors=blue]{hyperref}

\title{Taylor University\\
  Faculty Mentored Undergraduate\\
  Summer Scholarship Program\\
  Final Report}

\author{Tom Nurkkala and Ken Kiers}

\begin{document}

\maketitle

% From Sue Gavin --
% Faculty & Student reports need to include:
% What the objectives were
% How you approached / accomplished those objectives
% Next steps
% How the FMUS impacted your future plans
% Any other information you think is relevant to your work.

\section{Participants}
\label{sec:participants}

The co-principal investigators on the project were:
\begin{itemize}
\item Dr.\ Tom Nurkkala\\
  Associate Professor, Computer Science and Engineering\\
  \texttt{tnurkkala@cse.taylor.edu}
\item Dr.\ Ken Kiers\\
  Professor and Chair, Physics and Engineering\\
  \texttt{knkiers@taylor.edu}
\end{itemize}
Students involved in the project were:
\begin{itemize}
\item Keith Bauson, Computer Science, 2017
\item Abram Stamper, Computer Science, 2016
\end{itemize}

\section{Objectives}
\label{sec:objectives}

Our principal objective for this project was
to inaugurate long-term research and development
into the design, construction, and empirical evaluation
of a next-generation Learning Management System (LMS).

Current and emerging LMS's
suffer two key limitations.
First, little actionable information,
and second, emphasis on massive scalability
at the expense of
personalized student-teacher engagement.
Our project aimed to begin work to
design and construct
learning management software
that
(1)~\emph{provides actionable educational intelligence}
to support continuous, measurable improvements
in teacher productivity and student outcomes, and
(2)~\emph{personalizes the  teacher-student relationship}
into a modern apprenticeship model that
encourages mastery learning without sacrificing scalability.

Specifically, our proposed objectives
were as follows.
\begin{enumerate}
\item \textbf{Identify}
  important and compelling features of
  a next-generation LMS.
\item \textbf{Architect}
  a software architecture
  that provides a long-term platform
  on which to develop the new LMS.
\item \textbf{Build}
  and test features for the LMS.
\item \textbf{Deploy}
  completed features
  on Taylor servers
  for use in select fall classes.
\item \textbf{Involve}
  students in all aspects of the project.
\item \textbf{Mentor}
  students personally, professionally,
  and academically.
\item \textbf{Fund}
  ongoing learning management R\&D and student mentoring;
  collaborate with the Taylor Office of Sponsored Program
  to this end.
\end{enumerate}

\section{Results}
\label{sec:results}

Referring to the objectives enumerated in Section~\ref{sec:objectives},
we achieved the following results.

\subsection{Identify}
\label{sec:identify}

We already had a robust list set of features
based on past brainstorming.
Many of these features were suggested in our project proposal.

Our focus during this initial round of R\&D
was to create an ``MVP'' or \emph{minimum viable product}.
The goal of an MVP is to create a system
that has sufficient capability
to provide a meaningful set of services
to its end users (for us, students and teachers).
We judged that our MVP
must allow users to:
\begin{enumerate}
\item Schedule courses
\item Assign instructors to courses
\item Permit students to ``enroll'' in courses (in the system)
\item Allow students to register class attendance
\item Project content on the classroom projector
\item Evaluate student understanding using mobile devices
\item Support out-of-class assignments
\item Provide feedback on student work
\end{enumerate}

\subsection{Architect}

We devoted considerable effort
to an architecture that we anticipate
will support long-term R\&D in this area.
We:
\begin{enumerate}
\item Researched, identified, and selected
  leading-edge \emph{software, tools, and standards} for:
  \begin{enumerate}
  \item Web user interface
  \item Mobile user interface
  \item RESTful web services
  \item Secure access (authorization and authentication)
  \item Relational database management
  \end{enumerate}
  In all these architectural choices,
  we emphasized software and tools we judged to be
  robust, secure, scalable, and performant.
\item Devoted considerable attention, discussion, and revision
  to the \emph{relational data model} that we created
  to support the data persistence requirements
  of our LMS.
\item Designed a \emph{web services interface}
  that supports both web-based and mobile
  user software.
  Using state-of-the-art techniques for web services,
  this approach will allow us to provide access
  to the LMS from a wide variety of devices.
\end{enumerate}

\subsection{Build}

Of the MVP features listed in Section~\ref{sec:identify},
we either completed or made significant progress
on the first six.
Students were involved in all phases
of the software development life cycle.

To ensure the resulting software was robust,
we placed significant emphasis on
both unit testing and end-to-end testing.
Our tests cover nearly 100\% of the server software
and a large portion of the web interface.

\subsection{Deploy}

From the first day of the project,
we emphasized continuous delivery
of the LMS software as it was developed.
In that sense,
we achieved our deployment goal.

However,
because we did not complete all the MVP features
(Section~\ref{sec:identify}),
we did not reach the point at which
it made sense to deploy to Taylor servers.
Unfortunately,
this also means that we were not able to
deploy the software for trail use in the classroom
during the fall semester.

Consistent with our desire to make the software
available to others at no charge,
we have created a publicly-accessible software repository
containing the complete source code
for the system.
Code-named \emph{Faraday} after nineteenth-century
scientist, engineer, teacher, and Christ-follower Michael Faraday,
the system can be found at \url{https://github.com/faraday-effect/faraday}.

\subsection{Involve}

We worked directly with students
each week day
in the design, development, testing, and debugging
of the LMS.
Employng state-of-the-art
``agile'' software development
practices and principles,
our project emphasized
face-to-face communication,
flexible planning,
incremental development,
and continuously deliverable software.
We regularly employed ``pair programming''
in which a professor and a student
worked together on design, programming,
debugging, and testing.

\subsection{Mentor}

We created an environment that was
flexible, fun, and productive.
The project gave students an opportunity to interact
with us on a daily basis around
learning management,
software engineering,
and software engineering management.
The experience
helped prepare students
for future research and development work
as computer scientists and software engineers.

\subsection{Fund}

Although we intend to revisit this objective,
we were unable to research or secure
additional funding for ongoing work on the project.

\section{Future Work}

One summer's work only scratched the surface of what's
possible for the future of learning management.
Near term work in this area should:
\begin{enumerate}
\item \textbf{Complete} the implementation of the MVP features
  identified in Section~\ref{sec:identify}.
\item \textbf{Deploy} the system for trial use in the classroom.
\item \textbf{Study} the impact of new learning management
  techniques and technologies in the classroom.
\item \textbf{Refine} the system,
  identifying the most promising new features
  for implementation.
\end{enumerate}
Longer term plans should seek to:
\begin{enumerate}
\item \textbf{Collaborate}
  across departments on campus
  to explore the efficacy of the new learning management techniques
  in diverse disciplines.
\item \textbf{Disseminate} results
  with the scholarly community
  through publication of empirical results.
\item \textbf{Share} the software itself
  under a free and open-source software license that would
  allow other institutions to install and employ the
  system free of charge.
\item \textbf{Inaugurate}
  a global learning community around
  continuous improvement of
  teaching and learning
  through the judicious and sustainable application
  of actionable, personalized learning management technology.
\end{enumerate}

\end{document}

%%% Local Variables:
%%% mode: latex
%%% TeX-master: t
%%% End:
